% Chapter 1

\chapter{Background} % Main chapter title

\label{Chapter1} % For referencing the chapter elsewhere, use \ref{Chapter1} 

%----------------------------------------------------------------------------------------

% Define some commands to keep the formatting separated from the content 
\newcommand{\keyword}[1]{\textbf{#1}}
\newcommand{\tabhead}[1]{\textbf{#1}}
\newcommand{\code}[1]{\texttt{#1}}
\newcommand{\file}[1]{\texttt{\bfseries#1}}
\newcommand{\option}[1]{\texttt{\itshape#1}}

%----------------------------------------------------------------------------------------

\section{Introduction}
	The idea for this independent study came from multiple conversations by the authors regarding the lack of a parallel or cluster computing course at the university. Subsequently, we have realized the importance of large scale processing in the age of big data. Understanding and analyzing big sets of data is a necessity in an age where almost all electronic devices (large and small) are connected to a network and feeding in trillions of data points per second. 
	
	We decided to utilize a set of Raspberry Pi computers as a small-scale example of a cluster computing platform. Initially, we thought that utilizing commodity hardware such as servers for our cluster would be the way to go. Unfortunately, we do not have access to such hardware, but after further research, we concluded that the Raspberry Pi would be sufficient to demonstrate concepts in parallel and cluster computing. 
	

%----------------------------------------------------------------------------------------

\section{Raspberry Pi }

	The Raspberry Pi is a low-cost, single-board computer that is capable of running light-weight GNU/Linux operating systems such as Raspbian. The Raspberry Pi utilizes the system on a chip (SoC) architecture in order to integrate components of the computer such as the microprocessor, graphics processing unit, and WiFi device. The Raspberry Pi is fairly easy to configure as there is a large, open-source community willing to help with setup issues. Additionally, the Raspberry Pi Foundation has put in a lot of effort to provide detailed and up-to-date documentation.

\subsection{ARM Architecture}

	The Raspberry Pi central processing unit utilizes the ARM Architecture developed by the ARM Holdings. The ARM architecture paradigm is extremely relevant as there have been over 100 billion devices produced (as of 2017) that utilize the ARM instructuction set architecture. Most handheld devices including iPads and gaming consoles utilize ARM cores.
	
	ARM, also known as Advanced RISC (Reduced Instruction Set Computing) Machine requires less transistors than x86 processors (usually found in personal computers) which make it an attractive candidate for embedded systems seeking to lower costs and energy consumption on devices. The ARM architecture supported only a core-bit width of 32, but the newest version of ARM, ARMv8 now supports both 32 and 64 bits as of 2011. Note, the Raspberry Pi utilizes ARMv7. ARMv7 adheres to the load/store architecture which separates instructions into memory access and ALU (Arithmetic Logic Unit) operations. Memory access is simply the process of transfering data from the memory to registers. ALU operations consist of the actual operations on the loaded memory. 

\subsection{Raspberry Pi 3}
	The Raspberry Pi 3 Model B consists of a quad-core 64-bit CPU with one gigabyte of RAM in the system on a chip configuration. Additionally, the RPi 3 is equipped with wireless LAN and Bluetooth capabilities. It is important to note that while the WiFi interface is speedy enough for basic internet usage, it does cause considerable bottlenecks when utilizing a parallel processing interface. In fact, the 802.11n WiFi speeds were clocked around 45 Mbits/second, while the USB Gigabit LAN clocked around 320 Mbits/second. Configuring USB Gigabit LAN with a Raspberry Pi cluster does take a considerable amount of time to configure as MAC addresses and USB devices have to be specifically assigned in order to ensure a functional RPi cluster. 

\subsection{Raspberry Pi Zero W}
	The Raspberry Pi Zero W is a lighter-weight WiFi-enabled version of the RPi 3. It has a 1GHz,single-core CPU with 512 MB of RAM. We utilized four RPi Zero W computers as our slave nodes. These nodes were utilized in order to carry out parallel processing while the RPi 3 was utilized as the master node which sent out jobs using the message passing interface (MPI). 

%----------------------------------------------------------------------------------------

\section{Message Passing Interface (MPI)}

	MPI is a standardized and portable parallel computing library usually used on supercomputers. The standards for MPI were first developed as parallel programs were being written in C, C++, and FORTRAN.Today, many other languages such as Python, R, and Java use wrapper classes in order to implement message passing interfaces written in C++ and FORTRAN.

\subsection{mpi4py}
	mpi4py is a message passing interface library for the Python programming language. We decided to use the Python programming language as it offers the implementation of high-level data structures with a dynamic typing and binding paradigm. Additionally, mpi4py has become one of the most utilized parallel computing libraries, thus there exists much online documentation and assistance. For instance, the devleopment of Python libraries such as NumPy make it easy to implement sometimes complex data structures such as arrays and dataframes in a user-friendly manner.
	


%----------------------------------------------------------------------------------------

\section{Docker}



\subsection{References}



\subsubsection{A Note on bibtex}

The bibtex backend used in the template by default does not correctly handle unicode character encoding (i.e. "international" characters). You may see a warning about this in the compilation log and, if your references contain unicode characters, they may not show up correctly or at all. The solution to this is to use the biber backend instead of the outdated bibtex backend. This is done by finding this in \file{main.tex}: \option{backend=bibtex} and changing it to \option{backend=biber}. You will then need to delete all auxiliary BibTeX files and navigate to the template directory in your terminal (command prompt). Once there, simply type \code{biber main} and biber will compile your bibliography. You can then compile \file{main.tex} as normal and your bibliography will be updated. An alternative is to set up your LaTeX editor to compile with biber instead of bibtex, see \href{http://tex.stackexchange.com/questions/154751/biblatex-with-biber-configuring-my-editor-to-avoid-undefined-citations/}{here} for how to do this for various editors.

\subsection{Tables}

Tables are an important way of displaying your results, below is an example table which was generated with this code:

{\small
\begin{verbatim}
\begin{table}
\caption{The effects of treatments X and Y on the four groups studied.}
\label{tab:treatments}
\centering
\begin{tabular}{l l l}
\toprule
\tabhead{Groups} & \tabhead{Treatment X} & \tabhead{Treatment Y} \\
\midrule
1 & 0.2 & 0.8\\
2 & 0.17 & 0.7\\
3 & 0.24 & 0.75\\
4 & 0.68 & 0.3\\
\bottomrule\\
\end{tabular}
\end{table}
\end{verbatim}
}

\begin{table}
\caption{The effects of treatments X and Y on the four groups studied.}
\label{tab:treatments}
\centering
\begin{tabular}{l l l}
\toprule
\tabhead{Groups} & \tabhead{Treatment X} & \tabhead{Treatment Y} \\
\midrule
1 & 0.2 & 0.8\\
2 & 0.17 & 0.7\\
3 & 0.24 & 0.75\\
4 & 0.68 & 0.3\\
\bottomrule\\
\end{tabular}
\end{table}

You can reference tables with \verb|\ref{<label>}| where the label is defined within the table environment. See \file{Chapter1.tex} for an example of the label and citation (e.g. Table~\ref{tab:treatments}).

\subsection{Figures}

There will hopefully be many figures in your thesis (that should be placed in the \emph{Figures} folder). The way to insert figures into your thesis is to use a code template like this:
\begin{verbatim}
\begin{figure}
\centering
\includegraphics{Figures/Electron}
\decoRule
\caption[An Electron]{An electron (artist's impression).}
\label{fig:Electron}
\end{figure}
\end{verbatim}
Also look in the source file. Putting this code into the source file produces the picture of the electron that you can see in the figure below.

\begin{figure}[th]
\centering
\includegraphics{Figures/Electron}
\decoRule
\caption[An Electron]{An electron (artist's impression).}
\label{fig:Electron}
\end{figure}

Sometimes figures don't always appear where you write them in the source. The placement depends on how much space there is on the page for the figure. Sometimes there is not enough room to fit a figure directly where it should go (in relation to the text) and so \LaTeX{} puts it at the top of the next page. Positioning figures is the job of \LaTeX{} and so you should only worry about making them look good!

Figures usually should have captions just in case you need to refer to them (such as in Figure~\ref{fig:Electron}). The \verb|\caption| command contains two parts, the first part, inside the square brackets is the title that will appear in the \emph{List of Figures}, and so should be short. The second part in the curly brackets should contain the longer and more descriptive caption text.

The \verb|\decoRule| command is optional and simply puts an aesthetic horizontal line below the image. If you do this for one image, do it for all of them.

\LaTeX{} is capable of using images in pdf, jpg and png format.

\subsection{Typesetting mathematics}

If your thesis is going to contain heavy mathematical content, be sure that \LaTeX{} will make it look beautiful, even though it won't be able to solve the equations for you.

The \enquote{Not So Short Introduction to \LaTeX} (available on \href{http://www.ctan.org/tex-archive/info/lshort/english/lshort.pdf}{CTAN}) should tell you everything you need to know for most cases of typesetting mathematics. If you need more information, a much more thorough mathematical guide is available from the AMS called, \enquote{A Short Math Guide to \LaTeX} and can be downloaded from:
\url{ftp://ftp.ams.org/pub/tex/doc/amsmath/short-math-guide.pdf}

There are many different \LaTeX{} symbols to remember, luckily you can find the most common symbols in \href{http://ctan.org/pkg/comprehensive}{The Comprehensive \LaTeX~Symbol List}.

You can write an equation, which is automatically given an equation number by \LaTeX{} like this:
\begin{verbatim}
\begin{equation}
E = mc^{2}
\label{eqn:Einstein}
\end{equation}
\end{verbatim}

This will produce Einstein's famous energy-matter equivalence equation:
\begin{equation}
E = mc^{2}
\label{eqn:Einstein}
\end{equation}

All equations you write (which are not in the middle of paragraph text) are automatically given equation numbers by \LaTeX{}. If you don't want a particular equation numbered, use the unnumbered form:
\begin{verbatim}
\[ a^{2}=4 \]
\end{verbatim}

%----------------------------------------------------------------------------------------

\section{Sectioning and Subsectioning}

You should break your thesis up into nice, bite-sized sections and subsections. \LaTeX{} automatically builds a table of Contents by looking at all the \verb|\chapter{}|, \verb|\section{}|  and \verb|\subsection{}| commands you write in the source.

The Table of Contents should only list the sections to three (3) levels. A \verb|chapter{}| is level zero (0). A \verb|\section{}| is level one (1) and so a \verb|\subsection{}| is level two (2). In your thesis it is likely that you will even use a \verb|subsubsection{}|, which is level three (3). The depth to which the Table of Contents is formatted is set within \file{MastersDoctoralThesis.cls}. If you need this changed, you can do it in \file{main.tex}.

%----------------------------------------------------------------------------------------

\section{In Closing}

You have reached the end of this mini-guide. You can now rename or overwrite this pdf file and begin writing your own \file{Chapter1.tex} and the rest of your thesis. The easy work of setting up the structure and framework has been taken care of for you. It's now your job to fill it out!

Good luck and have lots of fun!

\begin{flushright}
Guide written by ---\\
Sunil Patel: \href{http://www.sunilpatel.co.uk}{www.sunilpatel.co.uk}\\
Vel: \href{http://www.LaTeXTemplates.com}{LaTeXTemplates.com}
\end{flushright}
